\chapter{Technologies used}

The application was created utilizing the capabilities of various powerful technologies such as
\begin{itemize}
	\item \textbf{Java Swing} for creating an interactive and user-friendly interface and a couple of libraries:
	
	\begin{itemize}
		\item \texttt{jxMapViewer2}
		
		\item \texttt{jxFreeChart}
	\end{itemize}
	
	\item the \textbf{MongoDB driver} for effectively managing and manipulating data stored in MongoDB
	\item the \textbf{Neo4j driver} for managing and querying graph data
	\item \textbf{Gson} for handling JSON data in a convenient and efficient manner
\end{itemize}
All of these technologies were integrated seamlessly to deliver a robust and high-performing application.

\paragraph{REST API}
Moreover,we makes use of the powerful Java standard library to implement a RESTful API using the built-in web server. The Java standard library provides a simple yet effective way to create and manage a web server, allowing for the creation of a lightweight and efficient REST API that adheres to industry standards. The use of the built-in web server eliminates the need for additional third-party dependencies, making the development process more streamlined and the deployment process more manageable. 

This choice of technology also ensures compatibility and ease of integration with other Java based systems. Additionally, Gson library is used for serialization and deserialization of Java data access objects (DAOs) to and from JSON, providing a convenient and efficient way to handle data transfer between the application and the API.


\begin{figure}[H]
	\centering
	\begin{subfigure}[b]{0.2\textwidth}
		\centering
		\includegraphics[width=\textwidth]{assets/Duke3D}
		\caption{}
	\end{subfigure}
	\hfill
	\begin{subfigure}[b]{0.4\textwidth}
		\centering
		\includegraphics[width=\textwidth]{assets/mongoDB_logo.png}
		\caption{}
	\end{subfigure}
	\hfill
	\begin{subfigure}[b]{0.3\textwidth}
		\centering
		\includegraphics[width=\textwidth]{assets/logo_neo4j.png}
		\caption{}
	\end{subfigure}
	\caption{Logo}
	\label{fig:routingsdiff}
\end{figure}


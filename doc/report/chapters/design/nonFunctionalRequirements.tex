\chapter{Not functional requirements}

\begin{itemize}
	\item To solve a routing problem in an acceptable amount of time
	
	\item A cache for queries to have a better response time for frequent 
	queries
	
	\item To find a good approximation of the optimal route, finding a good 
	balance between the optimal result and execution time of the procedure on 
	the server
	
	\item Respect the access constraints on the graph, such as oneway streets 
	and other access restrictions imposed by the highway's authority
	
	\item Use a routing algorithm that uses an heuristic function to compute a 
	path, such as \textit{A*}, to avoid excessive exploration of the hypothesis 
	space, 
	since those algorithms are always exponential in the graph's dimension
	
	\item Define a good heuristic to guide the algorithm on its visit on the 
	graph, considering features such as the road's classification, speed limit, 
	the geographical distance between the vertices.
	
	\item To ensure availability of the data and partition tolerance
	
	\item Define appropriate requests by the application to minimize the data 
	exchange between the client and the server over the network, in particular 
	to reduce latency of the application when retrieving the \textit{Point of 
	Interest}s
	
	\item Define appropriate auxiliary data structure to speed-up the queries 
	on the geographical data

	\item Define appropriate redundancies to allow the user to quickly access 
	information regarding the currently active disruptions. So we have to avoid 
	to access useless information from the archived data
	
	\item Ensure that in the event of a problem in the TIMS' logic or in the 
	\textit{Transport for London}'s \textit{API}s the active 
	disruptions will remain active after their programmed expiration date
\end{itemize}
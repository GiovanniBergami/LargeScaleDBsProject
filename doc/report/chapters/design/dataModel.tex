\chapter{Databases design}

%TODO INCLUDE PDF WHEN COMPILING
\includepdf[landscape=true,pagecommand=\thispagestyle{plain}]{assets/uml.pdf}

\chapter{Data model}

\paragraph{}
The data model section includes a description of the document collections, 
graph nodes and keys that are stored in the database. Two types of DBMSes have 
been selected for this purpose. 

\paragraph{Document DB}
MongoDB has been chosen as the database management system for the document 
database part, which stores information about the Point Of Interest and the 
concluded disruptions.

The decision to use a document database was based on its flexibility and 
ability to perform complex queries and works as a way to store an history of 
these informations.

\paragraph{Graph DB}
To manage the routing, where users can insert two points and the application will compute a travel route between them, a graph database managed using Neo4j has been selected to better support these features.

\section{Document database}

\paragraph{DBMS} Our choice for the \textit{database management system} to 
handle the document database was \textit{MongoDB}, since it is the most popular 
\textit{DBMS} of its kind and it also provides several functionalities useful 
for our use case, such as indexes and a powerful query engine.

\paragraph{Collections} In MongoDB we created the following collections:

\begin{itemize}
	\item POIs
	\item Disruptions
\end{itemize}

\subsubsection{Point of Interest}
The POIs collection looks like this:

\lstinputlisting{../../schema/example.poi.json}

Where the map \textit{tags} has not a particular structure, but it stores the 
\textit{OpenStreetMap}'s tag for that particular POI. It might store 
information such as its website, its opening hours, its physical dimensions, if 
it is accessible in a wheelchair etc...

\paragraph{Main fields}
\begin{itemize}
	\item \texttt{coordinates} a \textit{GeoJson} document of type 
	\texttt{Point} representing the geographical location of the object
	
	\item \texttt{name} the name of the object, if any
\end{itemize}

\subsubsection{Disruption}
The disruption collection is organized in the following structure:

\lstinputlisting{../../schema/example.disruption.json}

This representation might allow in the future to store additional optional 
values for certain kinds of POIs if the need arises, without any compatibility 
issue for the existing code; for example one might want to store a restaurant’s 
opening hours or the accessibility level for wheelchair users in a certain 
building.

\paragraph{Main fields}
\begin{itemize}
	\item \texttt{boundaries} a \textit{GeoJson} document that could be both of 
	class \texttt{Polygon} or \texttt{Multipolygon}, it represents the effected 
	area. It might not be present in all documents.
	
	\item \texttt{coordinates} the main point effected by the disruption, that 
	is where the application is supposed to draw the sign.
	
	\item \texttt{streets} a list of streets effected by the disruptions and if 
	are closed or not to the traffic
	
	\item \texttt{updates} an archive of updates that \textit{Transport for 
	London} has published
\end{itemize}

\section{Graph database}

\paragraph{DBMS} Our choice for the \textit{database management system} to 
handle the graph database was \textit{Neo4j}, since it can easly handle a huge amount of nodes.


\paragraph{Data} The graph DB is mainly used to store the road network of London and the current active disruption, this is needed so that the routing algorithm can avoid adding to the frontier nodes in an area affected by a closure or increase the weight of nodes in areas affected by critical disruptions.

\paragraph{Schema}For the map side of things we store one class of node and one class of relationship:

\begin{itemize}
	\item \textbf{Intersection }(due to a mistake it is actually called Point 
	in the database) represents the connection between one or more ways, it is 
	a \textit{vertex} of the graph
	
	\lstinputlisting{../../schema/example.point}
	
	\begin{itemize}
		\item \texttt{coord} Those are the geographical coordinates of the 
		node
	\end{itemize}
	
	\item \textbf{Connects} represents the connection between two Insersection. 
	It stores informations like its name and the cost of traversing it and 
	eventual access restrictions, like one way streets or motor-only roads
	
	\lstinputlisting{../../schema/example.connects}
	
	\begin{itemize}
		\item \texttt{maxspeed} It is the maximum speed on that stretch of road 
		expressed in meters per second
		
		\item \texttt{crossTime} It is the estimated crossing time of the edge 
		for each transportation mode. If the cross time is $\infty
		$ then it means 
		that the road has an access restriction in that direction, such has no 
		entry signs.
		
		It was used to use with the \textit{GDS}' implementation of 
		\textit{A*}, since we have implemented our version of \textit{A*} we 
		only check for $\infty$
		
		\item \texttt{class} The road class in the hierarchy where 
		\textit{motorway} is at the top.
	\end{itemize}
	
\end{itemize}

For the disruption handling we have the following nodes and relationships:

\begin{itemize}
	\item \textbf{Disruption} a node in the graph containing all the 
	informations about an active disruption
	
	\lstinputlisting{../../schema/example.disruption}
	
	We save in this edge all the information that could be of immediate help to 
	the user, to avoid making another query to the other databases when 
	requesting a preview of the data. In particular we store:
	
	\begin{itemize}
		\item \texttt{ttl} It is the \textit{Time to live}, that is the number 
		of seconds that remains to programmed end of the disruption, so that if 
		there are problems in the \textit{driver TIMS} the \textit{DBMS} can 
		drop it
	\end{itemize}
	
	\item \textbf{isDisrupted} is a relation between a Point and a Disruption, 
	telling that the road is being affected by a disruption.
	
	We don't store any information on those edges.

\end{itemize}
	
\begin{figure}[H]
	\centering
	\includegraphics[width=0.7\linewidth]{assets/schemaneo4j}
	\caption{A possible portion of graph}
	\label{fig:schemaneo4j}
\end{figure}

\paragraph{Example}
Now it follows a snapshot of the graph:

\includepdf[landscape=false,pagecommand=\thispagestyle{plain}]{assets/graph}



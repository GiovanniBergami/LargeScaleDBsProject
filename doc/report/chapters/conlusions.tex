\part{Conclusion}

\section{The Benefits of Using Graph and Document Databases in Transportation Systems}
In addition to meeting all objectives, the use of a graph database in our application has also brought several advantages. It has provided us with a highly flexible and scalable data storage solution, allowing us to easily add and modify data as needed. The graph structure of the database allows for efficient querying and indexing of data, enabling quick and accurate retrieval of information. This is particularly useful in the context of routing, as it allows for real-time updates to the routing information and the ability to quickly find alternative routes for users in the event of a disruption. We also used a document database to store the history of disruptions and Points of Interest (POIs).

By using a document database, we were able to store detailed information about each disruption and POI, including timestamps, location data, and any other relevant details. This allows us to maintain a historical record of disruptions and POIs, and provides us with valuable data for analyzing past disruptions and identifying patterns.

Furthermore, the use of a graph database also enables more advanced analytics and data visualization. The ability to create relationships between different data points and analyze them as a network provides a deeper understanding of the data and allows for more accurate predictions and recommendations. This can be particularly beneficial for the statistician, as it would enable them to make more informed decisions and proposals for improving traffic in London.

The use of both a graph database and a document database in our application allowed us to efficiently store and manage a wide variety of data, providing us with a comprehensive view of the transportation system and enabling more accurate predictions and recommendations for improving traffic in London.

\section{Possible expansions in the future}

In conclusion, the use of a graph and document database in our application has not only allowed us to meet our objectives but also brought several advantages such as efficient querying, advanced analytics, and scalability. It has opened up the possibility for more advanced functionalities to be implemented in the future, such as incorporating data on public transportation and the disruptions that it can cause, providing more comprehensive information and enabling more accurate predictions and recommendations for improving traffic in London.

\section{Possible query on public trasportations}

\textit{For each class of public transportation disruption, find the top 3  lines that are most affected.}

\begin{figure}[H]
	\begin{lstlisting}
	db.Disruption.aggregate([
	{
		$match: {
			typeDisruption: "PUBLIC_TRANSPORT"
		}
	},
	{
		$group: {
			_id: "$terminatedDisruption.category",
			count: { $sum: 1 },
			line: { $first: "$routes.line" }
		}
	},
	{
		$sort: { count: -1 }
	},
	{
		$limit: 3
	}
	])
	\end{lstlisting}
	\caption{Top 3 public transportation disruption in MongoDB query}
\end{figure}
\part{Implementation}

\chapter{Queries on graph database}

In this chapter are presented the relevant queries for the
Neo4j graph database.

\section{Routing}

Our routing query is used to compute a suboptimal path between two points on the map, given as input to the query, using the algorithm known as \textbf{Anytime A*}.

\begin{figure}[H]
	\lstinputlisting{../../queries/routing.cypher}
	\caption{Cypher query}
\end{figure} 

The procedure \textit{lodonSafeTravel.route.anytime} has been implemented as show below:

%\begin{figure}[H]
\lstinputlisting[linerange={50,163}]{../../routingNeo4jProcedure/src/main/java/londonSafeTravel/RoutingAStart.java}
%	\caption{implementation of the procedure}
%\end{figure} 

\section{Point finding}

\paragraph{}
To ensure that the user selects a reachable point in the network given the user's transportation mode, we use \textit{connects} relationships between points in the graph to reduce the probability of selecting an unreachable point.

\begin{figure}[H]
	\lstinputlisting{../../queries/nearest_point.cypher}
	\caption{Cypher query}
\end{figure} 

\paragraph{}
For example, we in the user selects a pathway in a park and \textit{motor vehicle} is selected as transportation mode, the query will return the node relative to the nearest road open to motor traffic.

\paragraph{}
As stated in the previous chapters, a restriction of access for a certain mode of transportation is represented in the graph as cross time of positive infinity.

\part{Implementation}

\chapter{Queries on graph database}

In this chapter are presented the relevant queries for the
Neo4j graph database.

\section{Routing}

Our routing query is used to compute a suboptimal path between two points on the map, given as input to the query, using the algorithm known as \textbf{Anytime A*}.

\begin{figure}[H]
	\lstinputlisting{../../queries/routing.cypher}
	\caption{Cypher query}
\end{figure} 

The procedure \textit{lodonSafeTravel.route.anytime} has been implemented as show below:

%\begin{figure}[H]
\lstinputlisting[linerange={50,163}, language=Java]
	{../../routingNeo4jProcedure/src/main/java/londonSafeTravel/RoutingAStart.java}
%	\caption{implementation of the procedure}
%\end{figure} 

\section{Point finding}

\paragraph{}
To ensure that the user selects a reachable point in the network given the user's transportation mode, we use \textit{connects} relationships between points in the graph to reduce the probability of selecting an unreachable point.

\begin{figure}[H]
	\lstinputlisting{../../queries/nearest_point.cypher}
	\caption{Cypher query}
\end{figure} 

\paragraph{}
For example, we in the user selects a pathway in a park and \textit{motor vehicle} is selected as transportation mode, the query will return the node relative to the nearest road open to motor traffic.

\paragraph{}
As stated in the previous chapters, a restriction of access for a certain mode of transportation is represented in the graph as cross time of positive infinity.


\section{DocumentDB CRUD operations}
Here below are reported the principal CRUD operations.

\paragraph{Disruption: Create and Update}
The insertion operation in the DocumentDB is performed by a TIMS API. At regular intervals, every 10 minutes, the API is called and returns a list of disruption updates. The Set operation inserts a new disruption if the returned ID is not already present in the DB or updates it otherwise.
\lstinputlisting[linerange={35-48}, language=Java]
{../../src/libCommon/src/main/java/londonSafeTravel/dbms/document/ManageDisruption.java}

\paragraph{Disruption: Read}
The Get method must return a disruption when is invoked, given its ID as input.
\lstinputlisting[linerange={31-33}, language=Java]
{../../src/libCommon/src/main/java/londonSafeTravel/dbms/document/ManageDisruption.java}

\paragraph{Point of interest: Create}
To make the service usable, it was necessary to enter the POIs in the document database. A POI is entered using the following function:

\lstinputlisting[linerange={79-81}, language=Java]
{../../src/libCommon/src/main/java/londonSafeTravel/dbms/document/PointOfInterestDAO.java}
% !TeX root = ../report.tex

\part{Feasibility}
\chapter{Data sources}

We’re combining data from two organizations:

\begin{itemize}
	\item Transport for London TIMS
%	\item Transport for London public disruptions API
	\item OpenStreetMap
\end{itemize}

\paragraph{Introduction}
At the very beginning, we made a feasibility study of the project idea, in 
order to identify both the pros and the critical aspects of the application, 
considering the time and technical constraints at the time and thinking how to 
possibly improve our work in future.

\section{Transport for London}

\paragraph{Scraping}
Transport for London provides data at regular intervals – every five 
minutes –  and in a standardized format; this kind of data, especially when 
scrapped during an adequate period of time can provide a good amount of 
information to be analyzed by our application. Those analyses become of 
particular interest when we consider that Transport for London provides 
data with a great amount of detail, including things such as minor collisions 
and broken traffic lights. In our particular case we collected data for circa a 
month using \textit{Transport for London}’s JSON API with a simple shell script 
that was being executed 
automatically by a \textit{SystemD} timer every ten minutes.

\paragraph{Format}
The data is provided in \textit{JSON} format; in particular it is a list of 
documents each of the documents represents an update of an active disruption.

%\paragraph{Schema}
%The main challenge was to find a ``schema'' for the document database to store
%in
%an unified way, that is to share the same MongoDB collection, between road
%disruptions and public transportation disruptions; as it could be useful for
%certain operations to have an unified view of such data.

\paragraph{Schema}
The main challenge was to find a ``schema'' for the document database to store
terminated disruption to make possible to execute fast analytics on them.

\paragraph{Sources}
It is possible to access the documentation of the \textit{API} from here:
\url{https://api-portal.tfl.gov.uk/}

\section{OpenStreetMap}

\paragraph{Scraping}
Since the map’s data would be used only for routing purposes, we don’t need all 
the information provided by OpenStreetMap; so, to reduce the dimension of the 
data and maintain only the useful information, we decided to do some pruning of 
the network graph, in particular we removed useless nodes such as buildings, 
waterways, trees, parks and so on.

\paragraph{Schema}
The main challenge was to find a good representation of OpenStreetMap’s data 
for our chosen graph database (Neo4j) because it has to represents facts like 
one ways streets, access restrictions and the elements' geographical positions;
we iterated over several possible schemas for the graph database and over many
different ways to import the raw data into it.

\paragraph{Format}
Data is provided in an \textit{XML} format, after some processing we obtained
two \textit{CSV}s one containing all the intersection in the network and the
other all the highway's stretches, that is the edges between intersections, in
the network.

\begin{figure}[H]
	\lstinputlisting[lastline=5]{../../src/libCommon/examples/nodes.sample.csv}
	\caption{Example of intersections}
\end{figure}

\begin{figure}[H]
	\lstinputlisting[lastline=5]{../../src/libCommon/examples/ways.sample.csv}
	\caption{Example of intersections}
\end{figure}

\paragraph{Sources}
We imported the map's data from an \textit{XML} file provided by Geofrabik from
here:
\url{https://download.geofabrik.de/europe/great-britain/england/greater-london.html}


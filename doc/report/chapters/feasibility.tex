% !TeX root = ../report.tex

\part{Feasibility}
\chapter{Data sources}

We’re combining data from three sources and two organizations:

\begin{itemize}
	\item Transport for London TIMS
	\item Transport for London public disruptions API
	\item OpenStreetMap
\end{itemize}

\paragraph{Introduction}
At the very beginning, we made a feasibility study of the project idea, in 
order to identify both the pros and the critical aspects of the application, 
considering the time and technical constraints at the time and thinking how to 
possibly improve our work in future.

\section{Transport for London}

\paragraph{Scraping}
Transport for London provides data at regular intervals – every five 
minutes –  and in a standardized format; this kind of data, especially when 
scrapped during an adequate period of time can provide a good amount of 
information to be analyzed by our application. Those analyses become of 
particular interest when we consider that Transport for London provides 
data with a great amount of detail, including things such as minor collisions 
and broken traffic lights. In our particular case we collected data for circa a 
month using TLF’s JSON API with a simple shell script that was being executed 
automatically by a SystemD timer every ten minutes.

\paragraph{schema}
The main challenge was to find a “schema” for the document database to store in 
an unified way, that is to share the same MongoDB collection, between road 
disruptions and public transportation disruptions; as it could be useful for 
certain operations to have an unified view of such data.

\section{OpenStreetMap}

\paragraph{Scraping}
Since the map’s data would be used only for routing purposes, we don’t need all 
the information provided by OpenStreetMap; so, to reduce the dimension of the 
data and maintain only the useful information, we decided to do some pruning of 
the network graph, in particular we removed useless nodes such as buildings, 
waterways, trees, parks and so on.

\paragraph{Schema}
The main challenge was to find a good representation of OpenStreetMap’s data 
for our chosen graph database (neo4j) because it has to represents facts like 
one ways streets, access restrictions and the elements’ geographical positions; 
we iterate over several possible schemas for the graph database and over many 
different ways to import the raw data into it.

\begin{figure}[H]
%	\lstinputlisting[lastline=5]{../../src/libCommon/examples/nodes.csv}
	\caption{Example of intersections}
\end{figure}


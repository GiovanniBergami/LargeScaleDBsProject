% !TeX root = ../\dfrac{numeratore}{denominatore}report.tex

\part{Design}

\paragraph{}
bla bla bla

\input{chapters/design/mainActors.tex}
\input{chapters/design/functionalReq.tex}
<<<<<<< Updated upstream
=======
The application will allow its users to perform the following tasks:

\section{Client}

\begin{itemize}
	\item \textbf{View the map}
	\item \textbf{Move the map in a direction}, with the typical drag 'n drop
	\item \textbf{Increase or decrease the map magnifying}, with either buttons 
	or the mouse's wheel
	\item \textbf{Search for a specific place}, the following method are 
	supported:
	\begin{itemize}
		\item With an address
		\item With a POI's name
		\item With WGS84 geographical coordinates
	\end{itemize}
	\item \textbf{Ask for directions} between two intersections on the map
	\begin{itemize}
		\item Via car
		\item Via bicycle
		\item On foot
	\end{itemize}
	\item \textbf{Refresh active disruptions' data} clicking on the refresh button
\end{itemize}
\section{TIMS}
	\begin{itemize}
	\item \textbf{Create/Update a disruption}
	\item \textbf{Close a disruption}
	
	\end{itemize}
\section{Statician}
\begin{itemize}
	\item \textbf{Access the analytics tools}
	\item \textbf{Display the disruptions' heatmap}
\end{itemize}
\chapter{Not functional requirements}

\begin{itemize}
	\item To solve a routing problem in an acceptable amount of time
	
	\item A cache for queries to have a better response time for frequent 
	queries
	
	\item To find a good approximation of the optimal route, finding a good 
	balance between the optimal result and execution time of the procedure on 
	the server
	
	\item Respect the access constraints on the graph, such as oneway streets 
	and other access restrictions imposed by the highway's authority
	
	\item Use a routing algorithm that uses an heuristic function to compute a 
	path, such as \textit{A*}, to avoid excessive exploration of the hypothesis 
	space, 
	since those algorithms are always exponential in the graph's dimension
	
	\item Define a good heuristic to guide the algorithm on its visit on the 
	graph, considering features such as the road's classification, speed limit, 
	the geographical distance between the vertices.
	
	\item To ensure availability of the data and partition tolerance
	
	\item Define appropriate requests by the application to minimize the data 
	exchange between the client and the server over the network, in particular 
	to reduce latency of the application when retrieving the \textit{Point of 
	Interest}s
	
	\item Define appropriate auxiliary data structure to speed-up the queries 
	on the geographical data

	\item Define appropriate redundancies to allow the user to quickly access 
	information regarding the currently active disruptions. So we have to avoid 
	to access useless information from the archived data
	
	\item Ensure that in the event of a problem in the TIMS' logic or in the 
	\textit{Transport for London}'s \textit{API}s the active 
	disruptions will remain active after their programmed expiration date
\end{itemize}
	\begin{itemize}
	\item To solve a routing problem in an acceptable amount of time
	\item A cache for queries to have a better response time for frequent queries
	\item To find a good approximation of the optimal route
	\item Respect the access constraints on the graph, such as oneway streets
	\item Use a routing algorithm that uses an heuristic function to compute a path, such as A*, to avoid excessive exploration of the hypothesis space, since those algorithms are always exponential in the graph’s dimension
	\item Define a good heuristic to guide the algorithm on its visit on the graph, considering features such as the road’s classification, speed limit, the geographical distance between the vertices.
	\end{itemize}
>>>>>>> Stashed changes
\chapter{CAP Theorem issue}

In order to optimize performance for the anticipated high volume of read operations, it is essential to prioritize both high availability and low latency in the design of this application. Additionally, it is crucial that the system remains functional in the event of a partition.According to the CAP theorem, the design of this application should prioritize \textbf{Availability (A)} and \textbf{Partition Tolerance (P)} over Consistency (C). This means that the application is more focused on maintaining access to the system and tolerating partitioning rather than ensuring complete consistency of data.
\begin{figure}[H]
	\centering
	\includegraphics[width=0.4\linewidth]{assets/CAP_Theorem_Venn_Diagram}
	\caption{CAP theorem Venn diagram}
	\label{fig:captheoremvenndiagram}
\end{figure}
\chapter{Use cases}
\begin{figure}[H]
	\centering
	\includegraphics[width=\linewidth]{"assets/Immagine 2023-01-23 103038"}
	\caption{Use case diagram: User}
	\label{fig:immagine-2023-01-23-103038}
\end{figure}
\begin{figure}
	\centering
	\includegraphics{assets/useCaseStatTIMS}
	\caption[Use case: TIMS and statician]{Use case: TIMS and statician}
	\label{fig:usecasestattims}
\end{figure}

\chapter{Databases design}

%TODO INCLUDE PDF WHEN COMPILING
%\includepdf[landscape=true,pagecommand=\thispagestyle{plain}]{assets/uml.pdf}

\chapter{Data model}

\paragraph{}
The data model section includes a description of the document collections, 
graph nodes and keys that are stored in the database. Two types of DBMSes have 
been selected for this purpose. 

\paragraph{Document DB}
MongoDB has been chosen as the database management system for the document 
database part, which stores information about the Point Of Interest and the 
concluded disruptions.

The decision to use a document database was based on its flexibility and 
ability to perform complex queries and works as a way to store an history of 
these informations.

\paragraph{Graph DB}
To manage the routing, where users can insert two points and the application will compute a travel route between them, a graph database managed using Neo4j has been selected to better support these features.

\section{Document database}

\paragraph{DBMS} Our choice for the \textit{database management system} to 
handle the document database was \textit{MongoDB}, since it is the most popular 
\textit{DBMS} of its kind and it also provides several functionalities useful 
for our use case, such as indexes and a powerful query engine.

\paragraph{Collections} In MongoDB we created the following collections:

\begin{itemize}
	\item POIs
	\item Disruptions
\end{itemize}

The POIs collection looks like this:
\begin{lstlisting}
{
	"_id":
	"name":
	"type":
	"coordinates": {
		"type": "Point",
		"coordinates": [0.0, 0.0] // lat 'n lon
	}
}
\end{lstlisting}

The disruption collection is organized in the following structure:
\begin{lstlisting}
{
	"idDisruption":
	"type":
	"startTime":
	"endTime":
	"coordinates": {
		"type": "Point",
		"coordinates": [0.0, 0.0] // lat 'n lon
	},
	"boundaries": {
		"type": "Polygon",
		"coordinates": [
		[
		[100.0, 0.0],
		[101.0, 0.0],
		[101.0, 1.0],
		[100.0, 1.0],
		[100.0, 0.0]
		]
		]
	}
	
	,
	
	"category":
	"subcategory":
	"severity":
	"updates": [{
		"startTime":
		"endTime":
		"message":
	},
	
	]
	"streets": [{
		"name":
		"closure":
	},
	{
		"name":
		"closure":
	},
	..
	]
	"closure":
	
}
\end{lstlisting}
This representation might allow in the future to store additional optional values for certain kinds of POIs if the need arises, without any compatibility issue for the existing code; for example one might want to store a restaurant’s opening hours or the accessibility level for wheelchair users in a certain building.

\section{Graph database}

\paragraph{DBMS} Our choice for the \textit{database management system} to 
handle the graph database was \textit{Neo4j}, since it can easly handle a huge amount of nodes.


\paragraph{Data} The graph DB is mainly used to store the road network of London and the current active disruption, this is needed so that the routing algorithm can avoid adding to the frontier nodes in an area affected by a closure or increase the weight of nodes in areas affected by critical disruptions.

\paragraph{Schema}For the map side of things we store one class of node and one class of relationship:

\begin{itemize}
	\item \textbf{Intersection }(due to a mistake it is actually called Point in the database) represents the connection between one or more ways, it is a \textit{vertex} of the graph
	\item \textbf{Connects} represents the connection between two Insersection. It stores informations like its name and the cost of traversing it and eventual access restrictions, like one way streets or motor-only roads
	
\end{itemize}

For the disruption handling we have the following nodes and relationships:

\begin{itemize}
	\item \textbf{Disruption} a node in the graph containing all the informations about an active disruption
	\item \textbf{isDisrupted} is a relation between a Point and a Disruption, telling that the road is being affected by a disruption

\end{itemize}
	
\begin{figure}[H]
	\centering
	\includegraphics[width=0.7\linewidth]{assets/schemaneo4j}
	\caption{A possible portion of graph}
	\label{fig:schemaneo4j}
\end{figure}


\chapter{Distributed database design}

The distributed database's design is suggested by requirements of the applications.

\section{Replica set}

\paragraph{MongoDB}In order to ensure that all servers have the same data and 
to improve performance by allowing multiple servers to process queries, and can 
also provide a level of fault tolerance, as data can be retrieved from a 
replica server if the primary server goes down we decided to have three virual 
replicas with on each one a MongoDB instance hosted.

\begin{figure}[H]
	\centering
	\includegraphics[width=0.8\linewidth]{assets/replicaServerReades}
	\caption{}
	\label{fig:replicaserverreades}
\end{figure}


\paragraph{Neo4J}Regarding the Neo4J part, it is present only on one replica.

\paragraph{Composition} The three virtual machines are kindly provided by the University of Pisa and the replica set is composed of a primary replica that acts as the server that takes client requests, and two secondaries which are the servers that keep copies of the primary's data.

\paragraph{}

\begin{table} [H]
	\begin{center}
\begin{tabular}{|c|c|c|c|}
	\hline
	Virtual Machine & IP address & Port & OS \\
	\hline \hline
	Replica-0& 172.16.5.43 & 27017 &Ubuntu  \\
	\hline
	Replica-1& 172.16.5.47 & 27017 &Ubuntu  \\
	\hline
	Replica-2& 172.16.5.42 & 27017 &Ubuntu  \\
	\hline
\end{tabular}
\end{center}
\caption{\label{demo-table}Virtual machines settings}
\end{table}

\section{Replica Configuration}
The configuration is shown below:

\begin{lstlisting}
rsconf = {
	_id: "londonSafeTravelSet",
	members: [
	{
		_id: 0,
	 	host: "172.16.5.43:27017",
	 	priority:1
 	},
	{
		_id: 1,
	 	host: "172.16.5.47:27017",
	  	priority:2
  	},
	{
		_id: 2,
		 host: "172.16.5.42:27017",
		 priority:5
	}]
};

rs.initiate(rsconf);
\end{lstlisting}

\paragraph{}
After careful consideration, it was determined that the virtual machine with 
the IP address 172.16.4.43 should be given the highest priority and will serve 
as the primary replica, unless any unforeseen issues arise. As previously noted 
in regards to the handling of the CAP theorem, the application in question has 
a high ratio of read to write operations. Thus, in order to guarantee high 
availability and protection against partitioning, it was decided to adopt the 
Eventual Consistency paradigm. It should be noted, however, that in the event 
of partitioning, there may be instances where data returned may not be the most 
recent version.

\begin{figure}[H]
	\centering
	\includegraphics[width=0.66\linewidth]{assets/replicaServerWrites}
	\caption{}
	\label{fig:replicaserverwrites}
\end{figure}


\section{Replica crash}

\paragraph{Crash}In the event of failure of the primary node, the 
responsibility of primary role will be transferred to one of the two secondary 
replicas. As priorities have been assigned to each replica, it has been 
predetermined which of the secondary replicas will assume the role of primary. 
In this specific implementation, should the primary node become unavailable, 
the virtual machine with the IP address 172.16.4.47 will be designated as the 
new primary node. 

\begin{figure}[H]
	\centering
	\includegraphics[width=0.88\linewidth]{assets/newPrimary}
	\caption{}
	\label{fig:newprimary}
\end{figure}



\chapter{Overall platform architecture}

The application has been developed utilizing the Java programming language and 
the Intellij integrated development environment. As outlined in the Data Model 
section, MongoDB and Neo4j were chosen as the NoSQL database management systems 
for data storage and management. It was previously stated that there are three 
replicas implemented for MongoDB, and a single replica for Neo4j.

\begin{figure}[H]
	\centering
	\includegraphics[width=0.66\textwidth]{assets/diagram0.png}
	\caption{Application architecture}
	\label{fig:diagram0}
\end{figure}



\chapter{Technologies used}

The application was created utilizing the capabilities of various powerful technologies such as
\begin{itemize}
	\item \textbf{Java Swing} for creating an interactive and user-friendly interface and a couple of libraries:
	
	\begin{itemize}
		\item \texttt{jxMapViewer2}
		
		\item \texttt{jxFreeChart}
	\end{itemize}
	
	\item the \textbf{MongoDB driver} for effectively managing and manipulating data stored in MongoDB
	\item the \textbf{Neo4j driver} for managing and querying graph data
	\item \textbf{Gson} for handling JSON data in a convenient and efficient manner
\end{itemize}
All of these technologies were integrated seamlessly to deliver a robust and high-performing application.

\paragraph{REST API}
Moreover,we makes use of the powerful Java standard library to implement a RESTful API using the built-in web server. The Java standard library provides a simple yet effective way to create and manage a web server, allowing for the creation of a lightweight and efficient REST API that adheres to industry standards. The use of the built-in web server eliminates the need for additional third-party dependencies, making the development process more streamlined and the deployment process more manageable. 

This choice of technology also ensures compatibility and ease of integration with other Java based systems. Additionally, Gson library is used for serialization and deserialization of Java data access objects (DAOs) to and from JSON, providing a convenient and efficient way to handle data transfer between the application and the API.


\begin{figure}[H]
	\centering
	\begin{subfigure}[b]{0.2\textwidth}
		\centering
		\includegraphics[width=\textwidth]{assets/Duke3D}
		\caption{}
	\end{subfigure}
	\hfill
	\begin{subfigure}[b]{0.4\textwidth}
		\centering
		\includegraphics[width=\textwidth]{assets/mongoDB_logo.png}
		\caption{}
	\end{subfigure}
	\hfill
	\begin{subfigure}[b]{0.3\textwidth}
		\centering
		\includegraphics[width=\textwidth]{assets/logo_neo4j.png}
		\caption{}
	\end{subfigure}
	\caption{Logo}
	\label{fig:routingsdiff}
\end{figure}

